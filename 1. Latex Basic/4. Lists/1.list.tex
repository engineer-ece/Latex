\documentclass{book}

\begin{document}

\renewcommand{\labelenumi}{\arabic{enumi}}
\renewcommand{\labelenumii}{\arabic{enumi}.\arabic{enumii}}
\renewcommand{\labelenumiii}{\arabic{enumi}.\arabic{enumii}.\arabic{enumiii}}
\renewcommand{\labelenumiv}{\arabic{enumi}.\arabic{enumii}.\arabic{enumiii}.\arabic{enumiv}}

\begin{enumerate}
\item Introduction
\item Examples of basic lists
  \begin{enumerate}
    \item The itemize environment for bulleted (unordered) lists
    \item The enumerate environment for numbered (ordered) lists
    \item The description environment
  \end{enumerate}
\item Changing the label of individual entries
\item Nested lists
      \begin{enumerate}
        \item Nested lists: label style
        \begin{enumerate}
          \item Nested enumerate lists: number format
          \item Nested itemize lists: bullet style
        \end{enumerate}
      \end{enumerate}
\item Customizing lists
   \begin{enumerate}
    \item Customizing lists: changing labels
    \begin{enumerate}
      \item Standard label-generation commands
      \item Counter variables for enumerate
      \item Examples: customizing labels of enumerate lists
            \begin{enumerate}
             \item Practical example
             \item Printing counter example
             \item Non-practical(fun) example             
             
            \end{enumerate}
    \end{enumerate}
    \item Customizing lists: changing the layout
          \begin{enumerate}
            \item LaTeX list parameters
            \item Creating a custom list using the list environment
          \end{enumerate}
    \item Using the enumitem package to modify and create lists
     \begin{enumerate}
        \item enumitem package to modify and create lists
        \item Modifying a standard list
        \item Creating a new list with enumitem
              \begin{enumerate}
              \item Using \textbackslash setlist to configure a custom list created with enumitem
              \end{enumerate}
        \item Lists for lawyers: nesting lists to an arbitrary depth.
        \item Custom bullets using the enumitem package and MetaPost
              \begin{enumerate}
                \item Example 1: auto-sizing bullets points 
                \item Example 2: funky custom bullets
              \end{enumerate}                      
\end{enumerate}         
    
    
   \end{enumerate}         
\item Other features of the enumitem package
\end{enumerate}

\vspace{10ex}

\noindent
\huge 1. Introduction

  \normalsize \setlength{\parindent}{0.5in}This article provides an introduction to typesetting,\& customizing
  
 \begin{itemize}
  \item the itemize - unordered list 
  \item the enumerate - ordered list
  \item the description - list of descriptions 
\end{itemize}  
\noindent
\huge 2. Examples of basic lists

\large 2.1 The itemize

 \normalsize Lists are easy to create:
      \begin{itemize}
      
        \item List entries start with the \verb|\item| command.
        \item Individual entries are indicated with a black dot, a so-called bullet.
        \item The text in the entries may be of any length
      \end{itemize}
      
      \large 2.2 The enumerate

 \normalsize Lists are easy to create:
      \begin{enumerate}
      
        \item List entries start with the \verb|\item| command.
        \item Individual entries are indicated with a black dot, a so-called bullet.
        \item The text in the entries may be of any length
      \end{enumerate}
      
      \large 2.3 The descriptionss

 \normalsize Lists are easy to create:
      \begin{enumerate}
 \item List entries start with the \verb|\item| command.
            
       \begin{description}
       \item[Step 1:] List entries start with the \verb|\item| command.
        \item[Step 2:] Individual entries are indicated with a black dot, a so-called bullet.
        \item[Step 3:] The text in the entries may be of any length
      \end{description}
      
      \end{enumerate}

\noindent
\huge 3. Changing the label of individual entries
      
      
      
        
      



\end{document}